%
% bertha-tex: Project skeleton for scientific writing in LaTeX.
%
% Copyright 2020 Michael Haider <michael.haider@tum.de>
%
% Licensed under the Apache License, Version 2.0 (the "License");
% you may not use this file except in compliance with the License.
% You may obtain a copy of the License at
%
%     http://www.apache.org/licenses/LICENSE-2.0
%
% Unless required by applicable law or agreed to in writing, software
% distributed under the License is distributed on an "AS IS" BASIS,
% WITHOUT WARRANTIES OR CONDITIONS OF ANY KIND, either express or implied.
% See the License for the specific language governing permissions and
% limitations under the License.

\documentclass{bertha-tex-article}

\title{bertha-tex: Project skeleton for scientific writing in \LaTeX{}}

\author[email=michael.haider@tum.de]{Michael~Haider}
\author{Michael~Riesch}
\author{Christian~Jirauschek}

\affiliation{Department of Electrical and Computer Engineering, %
  Technical University of Munich, Arcisstr.~21, 80333~Munich, Germany %
}

\begin{document}

\maketitle

\begin{abstract}
  This project skeleton aims to provide a clean and solid basis for
  scientific paper writing in \LaTeX{}. It enables the use of predefined
  templates such that scientists can easily contribute to different
  scientific journals or conferences that typically require to use their
  own specific templates. The skeleton might also be used as a starting
  point for more advanced thesis or book projects.
\end{abstract}

\section*{Feature Overview}

The project skeleton includes several features that facilitategood software
engineering practices~\cite{riesch_project_2020}:

\begin{itemize}
  \item Usage of a version control system (VCS) and an appropriate work flow
  \item Usage of a project management tool including issue tracking
  \item Automated build system using CMake
  \item Automated packaging and deployment
  \item Automatic code formatting checks using latexindent and cmake-format
\end{itemize}

\bibliographystyle{unsrt}
\bibliography{references}

\end{document}
